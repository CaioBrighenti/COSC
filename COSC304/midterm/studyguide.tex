\documentclass{article}
\usepackage{amsmath} %This allows me to use the align functionality.
                     %If you find yourself trying to replicate
                     %something you found online, ensure you're
                     %loading the necessary packages!
\usepackage{amsfonts}%Math font
\usepackage{graphicx}%For including graphics
\usepackage{hyperref}%For Hyperlinks
\usepackage{listings}
\usepackage{graphicx}
\usepackage{natbib}        %For the bibliography
\usepackage{tikz}
\bibliographystyle{apalike}%For the bibliography
\usepackage[margin=1.0in]{geometry}
\usepackage{float}
\DeclareMathOperator{\EX}{\mathbb{E}}
\begin{document}
%set the size of the graphs to fit nicely on a 8.5x11 sheet
\noindent \textbf{Caio Brighenti }\\
\noindent \textbf{COSC 304 - Computer Theory}\\%\\ gives you a new line
\noindent \textbf{Fall 2019}\\%\\ gives you a new line
\noindent \textbf{Midterm Study Guide}\vspace{1em}\\
	\section{Sets}
	\textbf{Sets} are collections of values of one type with no internal structure. Sets have no repeated items and order does not matter.
		\subsection{Set Operations}
			\begin{itemize}
				\item Union - $A \cup B = \{x : x \in A \text{ or } x \in B\}$
				\item Intersect - $ A \cap B = \{ x : x \in A \text{ and } x \in B\}$
				\item Concatenation - $AB = \{ ab : a \in A \text{ and } b \in B\}$
				\item Cartesian Product - $A \text{x} B  = \{(a,b) : a \in A \text{ and } b \in B\}$
				\item Complement - $A^c = \{x : x \notin A\}$
				\item Powerset - $P(A) = \{A' : A' \subset A\}$
			\end{itemize}
		\subsection{Finite and Infinite Sets}
			Set $A$ is \textbf{finite} if it can be put in 1-to-1 correspondence with an initial segment of $\mathbb{N}$, or is the empty set $\O$. More intuitively, a set is finite if we can list the elements of $A$ in finite time.
			\\\\ Set $A$ is \textbf{countably infinite} if it can be put in 1-to-1 correspondence with \emph{all} of $\mathbb{N}$. Intuitively, a procedure can be devised to list all elements, but this procedure will never finish.
			\\\\ Set $A$ is \textbf{uncountably infinite} if it is not countable.
		\subsection{Relations and Functions}
		A \textbf{function} can be defined as a binary operation between two sets that given an input in set $A$ produces an output in set $B$. There are three types of functions:
		\begin{enumerate}
			\item Total function: $A \rightarrow B$ means for every input $A$ there is exactly one output $B$
			\item Partial function: $A \rightharpoonup B$ means for every input $A$ there is at most one output $B$
			\item Multi function: $A$ --- $B$ means for every input $A$ there are 0, 1, or many outputs in $B$
		\end{enumerate}
		We also define the \textbf{identity function} such that for any type $A$, $f(A) = A$.
		\subsection{Cardinality of Sets}
		The cardinality of a set $A$, or $|A|$, is the total number of elements in $A$.
		\begin{itemize}
				\item Union - $|A \cup B| = |A| + |B| - |A \cap B|$
				\item Intersect - $ |A \cap B| = |A| + |B| - |A\cup B|$
				\item Concatenation - 
				\item Cartesian Product - $|A \text{x} B|  = |A| \text{x} |B|$
				\item Powerset - $P(A) = 2^{|A|}$
				\item Total function - $|A \rightarrow B| = |B|^{|A|}$
				\item Partial function - $|A \rightharpoonup B| = |B+1|^{|A|}$
				\item Multi function - $|A$ --- $B| = |P(B)|^{|A|}$
			\end{itemize}
	\section{Alphabets and Languages}
		\subsection{Definitions}
		\textbf{Alphabet} $A$ is a set of single characters. \textbf{Word} $w$ in alphabet $A$ is a string of characters from $A$. $A^*$ denotes the set of all words in alphabet $A$. This operation is called \emph{Kleene star}.
		\subsection{Regular Languages}
	\section{Finite Automata}
		\subsection{Definitions}
		\subsection{Regular Expressions to f.a.}
		\subsection{Simplifying}
		\subsection{Converting ndfa to dfa}
		\subsection{Finding Language of f.a.}
	

\end{document}
