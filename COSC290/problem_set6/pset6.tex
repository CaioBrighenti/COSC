\documentclass[titlepage]{article}
\usepackage{hw}  % leave this line

\psetAuthor{Caio Brighenti}  % replace with your name
\psetNumber{6}  % replace with pset number

\begin{document} \maketitle

\section{Problem 1: Equivalence classes}

% Use \verb|\subsection{}| to separate your answer into parts, such as...

\subsection{DLN 8.110}  % make a new part

$R_1$ is an equivalence relationship if and only if it is reflexive, symmetric, and transitive.
\begin{itemize}
\item \emph{Reflexivity}: If $A$ is an empty set, then the $A = A = \emptyset$, so $\langle A,A \rangle \in R_1$. If $A$ is not an empty set, then the greatest element in $A$ will equal the greatest element in $A$, because $A=A$, thus $\langle A,A \rangle \in R_1$. Therefore $R_1$ is reflexive.
\item \emph{Symmetry}: In the case that $A = \emptyset$ and $\langle A,B\rangle \in R_1$, then by the second condition $B = \emptyset$. Therefore, also by the second condition, $\langle B,A\rangle \in R_1$. In the case that $\langle A,B\rangle \in R_1$ and $A \neq \emptyset$, the greatest element in $A$ must equal the greatest element in $B$. This means that the greatest element of $B$ must equal the largest element in $A$, and so $\langle B,A\rangle \in R_1$, making $R_1$ symmetric.
\item \emph{Transitivity}: If $\langle A,B \rangle \in R_1$ and $\langle B,C \rangle \in R_1$, and $A = \emptyset$, then $A=B=\emptyset$, and thus $B=C=\emptyset$. Therefore, if $A$,$B$, and $C$ are all empty sets, then $A=C=\emptyset$, and so $\langle A,C \rangle \in R_1$. In the case that $A$ is not an empty set, then the greatest element in $A$ must equal the greatest element in $B$. Since $\langle B,C \rangle \in R_1$, then the greatest element in $B$ must equal the greatest element in $C$. Thus, the greatest element in $A$ must equal the greatest element in $C$, therefore $\langle A,C \rangle \in R_1$. This means that the relation must be transitive.
\end{itemize}
Since the relation $R_1$ is reflexive, symmetric, and transitive, it is an equivalence relationship. The equivalence clauses are as follows:
\begin{itemize}
\item $\{ \emptyset\} $
\item $\{ \{0\}\} $
\item $\{ \{1\}, \{0,1\}\} $
\item $\{ \{2\}, \{0,2\}, \{1,2\}, \{0,1,2\}\} $
\item $\{ \{3\}, \{0,3\}, \{1,3\}, \{2,3\}, \{0,1,3\}, \{0,2,3\}, \{1,2,3\}, \{0,1,2,3\}\} $
\end{itemize}

\subsection{DLN 8.111} 

$R_2$ is an equivalence relationship if and only if it is reflexive, symmetric, and transitive. $P = \mathcal{P}(\{0,1,2,3\})$. 
\begin{itemize}
\item \emph{Reflexivity}: The sum of the items in $A$ will always equal the sum of items in $A$. Thus, for every $A \in P, \langle A,A \rangle \in R_2$. Therefore, the relation is reflexive.
\item \emph{Symmetry}: If the sum of items in $A$ equals the sum of items in $B$, the reverse must also be true. Therefore, for every $A,B \in P$, if $\langle A,B \rangle \in R_2$, then $\langle B,A \rangle \in R_2$. This means that $R_2$ is symmetric. 
\item \emph{Transitivity}: If $\langle A,B \rangle \in R_2$, then the sum of elements in $A$ equals the sum of elements in $B$. If $\langle B,C \rangle \in R_2$, then the sum of elements in $B$ must equal the sum of elements in $C$. Thus, the sum of elements in all three sets ($A,B,$ and $C$) must be equal, giving us $\langle A,C \rangle \in R_2$. This means that the relation $R_2$ is transitive.
\end{itemize}
Since the relation $R_2$ is reflexive, symmetric, and transitive, it is an equivalence relationship. The equivalence clauses are as follows:
\begin{itemize}
\item $\{ \emptyset\} $
\item $\{ \{0\}\} $
\item $\{ \{1\}, \{0,1\}\} $
\item $\{ \{2\}, \{0,2\}\} $
\item $\{ \{3\}, \{0,3\}, \{1,2\}, \{0,1,2\}\} $
\item $\{ \{1,3\}, \{0,1,3\}\} $
\item $\{ \{2,3\}, \{0,2,3\}\} $
\item $\{ \{1,2,3\}, \{0,1,2,3\}\} $
\end{itemize}

\subsection{DLN 8.113} 

$R_4$ is an equivalence relationship if and only if it is reflexive, symmetric, and transitive. $P = \mathcal{P}(\{0,1,2,3\})$. 
\begin{itemize}
\item \emph{Reflexivity}: For any $A \in P$, all items present in $A$ are also present in $A$. However, if $A = \emptyset$, $A \cap A$ will result in $\emptyset$, and thus $\langle A,A\rangle \notin R_4$ by the definition of $R_4$. This means that $R_4$ is not reflexive, as there is an $A \in P$ that does not satisfy $\langle A,A\rangle \in R_4$.
\end{itemize}
As an equivalence relationship depends on reflexivity, symmetry, and transitivity, proving that one of these three properties is not met is enough to show the relationship is not an equivalence relationship. Therefore, since $R_4$ is not reflexive, it is not an equivalence relationship.

\section{Problem 2: DLN 8.84}

\textbf{Claim}: The claim is that for any relation $R \subseteq A \times A$ that is both irreflexive and transitive, $R$ must also be asymmetric. 

\begin{proof}
We will prove by assuming the claim is false and showing a contradiction.

\begin{itemize}
\item \emph{Given}: Assume a relation $R \subseteq A \times A$ is irreflexive, transitive, and not asymmetric.  
\item \emph{Want to show}: The assumption leads to a contradiction.
\end{itemize}
By the definition of asymmetry, since $R$ is not asymmetric there must be one pair $\langle a,b\rangle \in R$ such that $\langle b,a\rangle \in R$. By the definition of transitivity, if $\langle a,b \rangle \in R$, and $\langle b,a\rangle \in R$, then $\langle a,a\rangle \in R$. Thus, if $R$ is not asymmetric and transitive, then we must have $\langle a,a \rangle \in R$.
\\
\\
By the definition of irreflexivity, for every $a \in A, \langle a,a\rangle \notin R$. By the assumption, $R$ is not asymmetric and is transitive, thus $\langle a,a \rangle \in R$, and is also irreflexive, thus for every $a \in A, \langle a,a\rangle \notin R$. These two statements are in direct contradiction, therefore if $R$ is both irreflexive and transitive, it must be asymmetric, proving the claim.


\end{proof}

\section{Problem 3: DLN 8.130}

\textbf{Claim}: Let $P(k)$ if there are no non-trivial cycles of length $k$ in any relationship $R$ that is both transitive and antisymmetric. The claim is that $\forall k \in \mathrm{Z}^{\geq 2}: P(k)$. 

\begin{proof}
We will prove by direct proof using the definitions of non-trivial cycles, transitivity, and anti-symmetry.

The definition of a non-trivial cycle states that a cycle in relationship $R$ is a sequence of $k$ \emph{distinct} elements $a_0,a_1,...,a_{k-1} \in A$ where $\langle a_1, a_{(i+1)mod k} \rangle \in R$ for each $i \in \{0,1,...,k-1\}$. This means that the upper bound for $i$ is $k-1$. By using this definition, the scenario where $i=k-1$ is as follows.
$$\langle a_{i},a_{(i+1)mod(k)} \rangle \in R$$
$$\langle a_{k-1},a_{(k-1+1)mod(k)} \rangle \in R$$
$$\langle a_{k-1},a_{(k)mod(k)} \rangle \in R$$
By the definition of the mod operator, this can be simplified to:
$$\langle a_{k-1},a_0 \rangle \in R$$
This means that in any relationship $R$ that contains a non-trivial cycle, the last item in the sequence of that cycle ($a_{k-1}$) will always be paired with the first ($a_0$).
\\
\\
Again by the definition of non-trivial cycles, we have:
$$\langle a_{i},a_{(i+1)mod(k)} \rangle \in R$$
As the upper bound for $i$ is $k-1$, for all other values of $i$ we have $i < k-1$. This means that the expression $a_{(i+1)mod(k)}$, can be simplified to $a_{i+1}$, as since $i < k-1$, $i + 1$ must always be smaller than $k$. This is because $a$  $mod(b) = a$ for all $a < b$. Using this simplification, we have:
$$\langle a_{i},a_{i+1} \rangle \in R$$
This means that for every $i \in \{0,1,...k-1\}$, $i$ is paired with $i+1$. We can show inductively that this implies $\langle a_0, a_{k-1}\rangle \in R$ through the transitive property. Since we have that both
$$\langle a_{i},a_{i+1} \rangle \in R$$
$$\langle a_{i+1},a_{i+2} \rangle \in R$$
By the transitive property of relationship $R$, we therefore have:
$$\langle a_{i},a_{i+2} \rangle \in R$$
As every $i \in \{0,1,...,k-1\}$ is paired with $i+1$, this holds for all elements in the cycle, thus giving us that $\langle a_{0},a_{k-1} \rangle \in R$.
\\
We now have that both:
$$\langle a_0, a_{k-1}\rangle \in R$$
$$\langle a_{k-1},a_0 \rangle \in R$$
Since $R$ is antisymmetric, $a_0 = a_{k-1}$. However, the definition of non-trivial cycles holds that all elements in the cycle are distinct, therefore this \emph{cannot} be a non-trivial cycle. Thus, we have $P(k)$, proving the claim.
\end{proof}

\end{document}