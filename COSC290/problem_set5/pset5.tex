\documentclass[titlepage]{article}
\usepackage{hw}  % leave this line

\psetAuthor{Caio Brighenti}  % replace with your name
\psetNumber{5}  % replace with pset number

\begin{document} \maketitle

\section{Problem 1: $G_n$ and $F_n$}

\textbf{Claim}: Let $P(n) := G_n = F_{n+3} - 1$.  The claim is that $\forall n \in \mathrm{N}^{\geq -1}: P(n)$.

\begin{proof}
We will prove by strong induction on $n$.

\textbf{Base cases}: $n=-1$ and $n=0$.  For the first case, $G_{-1} = F_{2}-1$, and by the definition of $F_n$ and $G_n$, $F_2 = 1$ and $G_{-1} = 0$, giving us $0 = 1 - 1$, thus the case is true for $n=-1$. For the second case, $G_0 = F_3 - 1$, so $1 = 2 - 1$, thus the second case is also true.


\textbf{Inductive case}: Let $n \geq 1$.  We will show $[P(-1) \land P(0) \land ... \land P(n)] \implies P(n+1)$.
\begin{itemize}
\item \emph{Given}: Assume $P(-1) \land ... \land P(n-1) \land P(n)$ is true.
\item \emph{Want to show}: $P(n+1)$ is true.
\end{itemize}
Since $P(n-1)$ is true, we have
$G_{n} = F_{n+2} - 1$.
Since $P(n)$ is true, we have
$G_{n-1} = F_{n+3} - 1$.

We will use this fact to show $P(n+1)$:
\begin{align*}
P(n) := G_{n} &= F_{n+3} - 1 & \text{definition of P(n)} \\
P(n+1) := G_{n+1} &= F_{n+4} - 1 & \text{inductive hypothesis} \\
G_{n+1} &= G_n + G_{n-1} + 1 & \text{definition of G} \\
F_{n+4} &= F_{n+3} + F_{n+2} & \text{definition of F} \\
P(n+1) := G_n + G_{n-1}&= F_{n+3} + F_{n+2} - 2 & \text{substituting/rearranging terms} \\
\end{align*}
By the claim, the following are given to be true:\\
\begin{align*}
P(n) := G_{n} = F_{n+3} - 1 \\ 
P(n-1) := G_{n-1} = F_{n+2} - 1
\end{align*}
By adding the two, we get:\\
\begin{align*}
G_n + G_{n-1}= F_{n+3} + F_{n+2} - 2
\end{align*}
This is exactly equal to the rearranged inductive hypothesis for $P(n+1)$. Thus, the inductive hypothesis must be true.
\end{proof}

\section{Problem 2: $G_n$ lower bound}

\textbf{Claim}: Let $P_2(h) := G_h \geq 2^{h/2}$.  The claim is that $\forall n \in \mathrm{N}^{\geq 0}: P_2(h)$.

\begin{proof}
We will prove by strong induction on $h$.

\textbf{Base cases}: $h=0$ and $h=1$.  For the first case, $G_{0} = 1$, and $2^{0/2} = 1$, giving us $1 \geq 1$ and thus the first case is true. For the second case, $G_1 = 2$, and $2^{1/2} = \sqrt{2}$, giving us $2 \geq \sqrt{2}$, thus the second case is also true. 


\textbf{Inductive case}: Let $h \geq 2$.  We will show $[P_2(0) \land P_2(1) \land ... \land P_2(h-2) \land P_2(h-1)] \implies P_2(h)$.
\begin{itemize}
\item \emph{Given}: Assume $P_2(-1) \land ... \land P_2(h-2) \land P_2(h-1)$ is true.
\item \emph{Want to show}: $P_2(h)$ is true.
\end{itemize}
Since $P_2(h-2)$ is true, we have
$G_{h-2} \geq 2^{(h-2)/2}$.
Since $P_2(h-1)$ is true, we have
$G_{h-1} \geq 2^{(h-1)/2}$.

We will use this fact to show $P_2(h)$:
\begin{align*}
P_2(h) := G_{h} &\geq 2^{h/2} & \text{definition of $P_2$(h)} \\
G_{h} &= G_{h-2} + G_{h-1} + 1 & \text{definition of G} \\
\end{align*}
By substituting $G_{h-2}$ and $G_{h-1}$ with the given cases, we get:\\
\begin{align*}
G_h \geq 2^{(h-2)/2} + 2^{(h-1)/2} + 1
\end{align*}
\end{proof}

\section{Problem 3: lower bound on height balanced binary trees}

\section{Problem 4: false lower bound on binary trees}

\section{Problem 5: bound on height}

\section{Example of proof by induction}

\emph{This is an example proof, provided in LaTeX so that you may copy its basic formatting.}

\textbf{Claim}: Let $P(n) := \sum_{i=1}^n i = \frac{n(n+1)}{2}$.  The claim is that $\forall n \in \mathrm{Z}^{\geq 1}: P(n)$.

\begin{proof}
We will prove by weak induction on $n$.

\textbf{Base case}: $n=1$.  In this case $\sum_{i=1}^n i = \sum_{i=1}^1 i = 1$ and $\frac{n(n+1)}{2} = \frac{1 \times (1+1)}{2} = 1$.  Thus $P(1)$ is true.


\textbf{Inductive case}: Let $n \geq 2$.  We will show $P(n-1) \implies P(n)$.
\begin{itemize}
\item \emph{Given}: Assume $P(n-1)$ is true.
\item \emph{Want to show}: $P(n)$ is true.
\end{itemize}
Since $P(n-1)$ is true, we have
$$\sum_{i=1}^{n-1} i = \frac{(n-1)((n-1)+1)}{2}$$

We will use this fact to show $P(n)$:
\begin{align*}
\sum_{i=1}^{n} i &= \left( \sum_{i=1}^{n-1} i \right) + n & \text{definition of summation} \\
&= \frac{(n-1)((n-1)+1)}{2} + n & \text{inductive hypothesis} \\
&= \frac{(n-1)n + 2n}{2} & \text{rearanging/simplifying terms} \\
&= \frac{n^2 - n + 2n}{2} = \frac{n^2 + n}{2}  & \text{algebra} \\
&= \frac{n(n + 1)}{2}  & \square
\end{align*}
\end{proof}

\end{document}