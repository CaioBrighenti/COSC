\documentclass[titlepage]{article}
\usepackage{hw}  % leave this line

\psetAuthor{Caio Brighenti}  % replace with your name
\psetNumber{5}  % replace with pset number

\begin{document} \maketitle

\section{Problem 1: $G_n$ and $F_n$}

\textbf{Claim}: Let $P(n) := G_n = F_{n+3} - 1$.  The claim is that $\forall n \in \mathrm{N}^{\geq -1}: P(n)$.

\begin{proof}
We will prove by strong induction on $n$.

\textbf{Base cases}: $n=-1$ and $n=0$.  For the first case, $G_{-1} = F_{2}-1$, and by the definition of $F_n$ and $G_n$, $F_2 = 1$ and $G_{-1} = 0$, giving us $0 = 1 - 1$, thus the case is true for $n=-1$. For the second case, $G_0 = F_3 - 1$, so $1 = 2 - 1$, thus the second case is also true.


\textbf{Inductive case}: Let $n \geq 1$.  We will show $P(-1) \land ... \land P(n-1) \land P(n) \implies P(n+1)$.
\begin{itemize}
\item \emph{Given}: Assume $P(-1) \land ... \land P(n-1) \land P(n)$ is true.
\item \emph{Want to show}: $P(n+1)$ is true.
\end{itemize}
Since $P(n-1)$ is true, we have
$G_{n} = F_{n+2} - 1$.
Since $P(n)$ is true, we have
$G_{n-1} = F_{n+3} - 1$.

We will use this fact to show $P(n+1)$:
\begin{align*}
P(n) := G_{n} &= F_{n+3} - 1 & \text{definition of P(n)} \\
P(n+1) := G_{n+1} &= F_{n+4} - 1 & \text{inductive hypothesis} \\
G_{n+1} &= G_n + G_{n-1} + 1 & \text{definition of G} \\
F_{n+4} &= F_{n+3} + F_{n+2} & \text{definition of F} \\
P(n+1) := G_n + G_{n-1}&= F_{n+3} + F_{n+2} - 2 & \text{substituting/rearranging terms} \\
\end{align*}
By the claim, the following are given to be true:\\
\begin{align*}
G_{n} = F_{n+3} - 1 \\ 
G_{n-1} = F_{n+2} - 1
\end{align*}
By adding the two, we get:\\
\begin{align*}
G_n + G_{n-1}= F_{n+3} + F_{n+2} - 2
\end{align*}
This is exactly equal to the rearranged inductive hypothesis for $P(n+1)$. Thus, the inductive hypothesis must be true.
\end{proof}

\section{Problem 2: $G_n$ lower bound}

\textbf{Claim}: Let $P_2(h) := G_h \geq 2^{h/2}$.  The claim is that $\forall h \in \mathrm{N}^{\geq 0}: P_2(h)$.

\begin{proof}
We will prove by strong induction on $h$.

\textbf{Base cases}: $h=0$ and $h=1$.  For the first case, $G_{0} = 1$, and $2^{0/2} = 1$, giving us $1 \geq 1$ and thus the first case is true. For the second case, $G_1 = 2$, and $2^{1/2} = \sqrt{2}$, giving us $2 \geq \sqrt{2}$, thus the second case is also true. 


\textbf{Inductive case}: Let $h \geq 2$.  We will show $[P_2(0) \land P_2(1) \land ... \land P_2(h-2) \land P_2(h-1)] \implies P_2(h)$.
\begin{itemize}
\item \emph{Given}: Assume $P_2(-1) \land ... \land P_2(h-2) \land P_2(h-1)$ is true.
\item \emph{Want to show}: $P_2(h)$ is true.
\end{itemize}
Since $P_2(h-2)$ is true, we have
$G_{h-2} \geq 2^{(h-2)/2}$.
Since $P_2(h-1)$ is true, we have
$G_{h-1} \geq 2^{(h-1)/2}$.

We will use this fact to show $P_2(h)$:
\begin{align*}
P_2(h) := G_{h} &\geq 2^{h/2} & \text{definition of $P_2$(h)} \\
G_{h} &= G_{h-2} + G_{h-1} + 1 & \text{definition of G} \\
\end{align*}
By substituting $G_{h-2}$ and $G_{h-1}$ with the given cases, we get:\\
\begin{align*}
G_h \geq 2^{(h-2)/2} + 2^{(h-1)/2} + 1
\end{align*}

It also follows that $2^{(h-2)/2} + 2^{(h-1)/2} + 1 > 2^{h/2}$, for all $h \geq 1$. This is because $2^{(h-2)/2}$ is equal to exactly half of $2^{h/2}$, and because $2^{(h-1)/2} > 2^{(h-2)/2}$. Thus $2^{h/2}$ must be smaller than $2^{(h-2)/2} + 2^{(h-1)/2} + 1$, as it is a number exactly half of  $2^{h/2}$ being added to a number bigger than half of $2^{h/2}$, plus one. Thus, if:
\begin{align*}
2^{(h-2)/2} + 2^{(h-1)/2} + 1 > 2^{h/2}
\end{align*}
And:
\begin{align*}
G_h \geq 2^{(h-2)/2} + 2^{(h-1)/2} + 1
\end{align*}
By the transitive property it follows that:
\begin{align*}
G_{h} \geq 2^{h/2}
\end{align*}
Thus, the claim must be true.
\end{proof}
\section{Problem 3: lower bound on height balanced binary trees}

\textbf{Claim}: For any height balanced binary tree $T$, $nodes(T) \geq G_{h(T)}$.

\begin{proof}
We will prove by structural induction on $h(T)$.

\textbf{Base cases}: $n=1$.  There are two base cases. The first base case is an empty tree. For an empty tree, $h(T) = -1$, and $nodes(T)=0$. Thus, $G_{h(t)}=0$, and since $0\geq0$, the claim is true for this base case. The second base case is a tree with a single node, and two empty subtrees. For this tree, $h(T) = 1 + max\{h(T_l), h(T_r)\} = 1 + -1 = 0$, and so $G_{h(T)} = 1$. In this case, $nodes(T)=1$, and since $1\geq1$, the claim is also true for this base case.


\textbf{Inductive case}: The inductive case is a height balanced binary tree $T$ that contains \textit{at least} one non-empty subtree.
\begin{itemize}
\item \emph{Given}: Assume the claim is true for the subtrees $T_l$ and $T_r$ of tree $T$.
\item \emph{Want to show}: The claim must be true for T.
\end{itemize}
By the definition of a binary tree, we have
$$nodes(T) = 1 + nodes(T_l) + nodes(T_r)$$
and,
$$h(T) = 1 + max\{h(T_l) + h(T_r)\}$$
By rearranging the second, we have
$$max\{h(T_l) + h(T_r)\} = h(T) - 1$$
Since the tree is height balanced, $h(Tl)$ and $h(Tr)$ differ by at most 1, thus they can both be equal to each other, or the smaller of the two must be $max\{h(T_l) + h(T_r)\} -1 = h(T) - 2$. We will first focus on the case where the height of the subtrees are \textit{not} equal. It is irrelevant which one of the two the larger one. Using these terms, we can rearrange the definition of $G_h$ to have the following
$$G_{h(T)}= G_{h(T)-2} + G_{h(T)-1} + 1$$
$$G_{h(T)}= G_{h(T_l)} + G_{h(T_r)} + 1$$\\

We will use this fact to show that the claim is true for T:
\begin{align*}
nodes(T) &\geq G_{h(T)} & \text{claim} \\
nodes(T) &\geq G_{h(T_l)} + G_{h(T_r)} + 1 & \text{substituting terms} \\
1 + nodes(T_l) + nodes(T_r) &\geq G_{h(T_l)} + G_{h(T_r)} + 1 & \text{definition of $nodes(T)$}\\
nodes(T_l) + nodes(T_r) &\geq G_{h(T_l)} + G_{h(T_r)}& \text{algebra}
\end{align*}
By the assumption, the claim is true for $T_l$ and $T_r$, thus
$$nodes(T_l) \geq G_{h(T_l)}$$
and
$$nodes(T_r) \geq G_{h(T_r)}$$
are both true. By adding both sides these, we have
\end{proof}
$$nodes(T_l) + nodes(T_r) \geq G_{h(T_l)} + G_{h(T_r)}$$
This is exactly equal to the rearranged claim for $T$ above. Thus, if the claim is true for the subtrees of $T$ and $T$ is height balanced, the claim must be true for $T$. There is also the case where $h(T_l) = h(T_r)$. In this scenario, the claim $nodes(T) \geq G_{h(T)}$ would still be true, as this change would only increase the left hand side of the expression. As the right hand side is a function of $h(T)$, which is defined by the maximum of the two subtree heights, it is irrelevant whether they are equal or one is smaller.

\section{Problem 4: false lower bound on binary trees}

In order to disprove the claim that for any binary tree $T$, $nodes(T)\geq G_{h(T)}$ we must simply provide a counterexample where the claim is false. A simple example of this is a scenario where the right subtree has a height of 2, and the left subtree is empty. We have
$$h(T_l) = -1$$
and
$$h(T_r) = 2$$
The height of this tree $T$ would be defined as
\begin{align*}
h(T) &= 1 + max\{h(T_l), h(T_r)\}  & \text{definition of $h(T)$} \\
&= 1 + 2 & \text{substituting terms} \\
&= 3 & \text{algebra} \\
\end{align*}
For the nodes, we would have
$$nodes(T_l) = 0$$
and
$$nodes(T_r) = 3$$
Since $T_l$ is empty, it must have 0 nodes. For $T_r$, there are multiple valid amounts of nodes. This specific tree has 3 nodes in the subtree $T_r$. The total nodes for $T$ would be
\begin{align*}
nodes(T) &= 1 + nodes(T_l) + nodes(T_r)  & \text{definition of $nodes(T)$} \\
&= 1 + 0 + 3 & \text{substituting terms} \\
&= 4 & \text{algebra} \\
\end{align*}
By the definition of $G_h$, and by substituting in the value above for $h(T)$, we have
$$G_{h(T)} = G_3 = 7$$
We will use these facts to show that the claim is not true for this $T$:
\begin{align*}
nodes(T) &\geq G_{h(T)}  & \text{claim} \\
4 &\geq 7 & \text{substituting terms} \\
\end{align*}
This statement is clearly false, as 4 is \textit{not} greater than or equal to 7. The claim is then false for this tree $T$, and thus disproven.
\section{Problem 5: bound on height}

\textbf{Claim}: The claim is that any non-empty height balanced tree with $n$ nodes has a height of at most $2log_2n$. In mathematical terms this is represented as $2log2n \geq h(T)$, where $n = nodes(T)$ and $T$ is a non-empty height balanced binary tree.

\begin{proof}
We will prove by direct proof.\\

By the previous proofs on $G_h$ and on height balanced binary trees, we have that
$$n \geq G_{h(T)}$$
And
$$G_h \geq 2^{h/2}$$
We can use these facts to prove the claim
\begin{align*}
G_h(T) &\geq 2^{(h(T)/2)}  & \text{substituting $h(T)$ in for $h$} \\
n &\geq 2^{(h(T)/2)} & \text{transitive property} \\
log_2(n) &\geq h(T)/2 & \text{algebra} \\
2log_2(n) &\geq h(T) & \text{algebra} \\
\end{align*}
This is exactly equal to the claim, thus given the previous proofs and the definition of $G_h$, the claim must be true.

\end{proof}
\end{document}