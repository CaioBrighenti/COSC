\documentclass[titlepage]{article}
\usepackage{hw}  % leave this line

\psetAuthor{Caio Brighenti}  % replace with your name
\psetNumber{7}  % replace with pset number

\begin{document} \maketitle

\section{Problem 1: Using induction to prove algorithm correctness, DLN 5.71}

\textbf{Claim}: Let $P(n)$ if for a sorted array $A[1...n]$ of length $n$, $binarySearch(A, x) \iff x \in A$ .  The claim is that $\forall n \in \mathrm{Z}^{\geq 0}: P(n)$.

\begin{proof}
We will prove by strong induction on $n$.

\textbf{Base cases}: The base cases are $n=0$ and $n=1$. Both $P(0)$ and $P(1)$ are true, as any array of length 0 or 1 is sorted.


\textbf{Inductive case}: Let $n \geq 2$.  We will show $P(n-1) \implies P(n)$.
\begin{itemize}
\item \emph{Given}: Assume $P(n-1)$ is true.
\item \emph{Want to show}: $P(n)$ is true.
\end{itemize}
Since $P(n-1)$ is true, we have
$$\sum_{i=1}^{n-1} i = \frac{(n-1)((n-1)+1)}{2}$$

We will use this fact to show $P(n)$:
\begin{align*}
\sum_{i=1}^{n} i &= \left( \sum_{i=1}^{n-1} i \right) + n & \text{definition of summation} \\
&= \frac{(n-1)((n-1)+1)}{2} + n & \text{inductive hypothesis} \\
&= \frac{(n-1)n + 2n}{2} & \text{rearanging/simplifying terms} \\
&= \frac{n^2 - n + 2n}{2} = \frac{n^2 + n}{2}  & \text{algebra} \\
&= \frac{n(n + 1)}{2}  & \square
\end{align*}
\end{proof}

\section{Problem 2: proving a relation is a partial order}

\subsection{DLN 8.131}

\emph{replace with your answer}

\subsection{DLN 8.132}

\emph{replace with your answer}

\section{Problem 3: an equivalence relation and a partial order? DLN 8.155}

\emph{replace with your answer}

\end{document}