\documentclass[titlepage]{article}
\usepackage{hw}  % leave this line

\psetAuthor{Caio Brighenti}  % replace with your name
\psetNumber{1}  % replace with pset number

\begin{document} \maketitle

\section{Problem 1}

\subsection{}  % make a new part

\(T = \{2, 3, 0\}\)

\subsection{}  % make a new part

\(P(A) = \{\{\},\{a\},\{b\},\{c\},\{a,b\},\{a,c\},\{b,c\},\{a,b,c\}\}\)

\subsection{}  % make a new part

\( bitstrs = \{\langle0,0,0\rangle,\langle0,0,1\rangle,\langle0,1,0\rangle,\langle0,1,1\rangle,\langle1,0,0\rangle,\langle1,0,1\rangle,\langle1,1,0\rangle,\langle1,1,1\rangle\} \)


\section{Problem 2}

\subsection{}
To map the elements of $P(A)$ using $bitstrs$, we can define a function $f$ that treats the values 0,1 as indications of whether an element is present in the set. For example, with the sequences \(\langle0,0,1\rangle\) the 0s in the first two positions indicate an absence of elements $a$ and $b$ in the set, while the 1 in the final position indicates the presence of $c$, thus resulting in the set \(\{c\}\). This mapping should iterate through each bit string sequence. The table below shows each of the $x$,$y$ mappings.

\begin{center}
\begin{tabular}{c|c}
\textbf{$x$} & \textbf{$y$} \\ \hline
\(\langle0,0,0\rangle\) & \(\{ \}\) \\
\(\langle0,0,1\rangle\) & \(\{c\}\) \\
\(\langle0,1,0\rangle\) & \(\{b\}\) \\
\(\langle0,1,1\rangle\) & \(\{b,c\}\) \\
\(\langle1,0,0\rangle\) & \(\{a\}\) \\
\(\langle1,0,1\rangle\) & \(\{a,c\}\) \\
\(\langle1,1,0\rangle\) & \(\{a,b\}\) \\
\(\langle1,1,1\rangle\) & \(\{a,b,c\}\) \\
\end{tabular}
\end{center}

\subsection{}

To iteratively generate the power set $P(S)$ where $S$ is a set of size $n$, we can use the sequences produced by $\{0,1\}^n$. Each sequence will correspond to exactly one set. Each set is generated by iterating through the corresponding sequence, represented by \(\langle x_1,x_2,...,x_n\rangle\). Each element $x_i$ will either be 0 or 1. Each element of the set $S$ is represented by $s_j$ for the $j^{th}$ element in S. $x_i$ will indicate the presence of the element $s_i$ in the set being produced, where 0 corresponds to the element \textit{not} being present and 1 corresponds to the element existing in the set. 

For example, in the set $A = {1,2,3}$, the bit string sequence \(\langle0,1,0\rangle\) will correspond to $x_1$ and $x_3$ being equal to 0, and $x_2$ being equal to 1. Thus, elements $s_1$ and $s_3$ will not be present, while $s_2$ will, resulting in the set \(\{2\}\).

\section{Problem 3}

\subsection{}

\(P(\emptyset) = \{\emptyset\}\)

\subsection{}
\(P(A) = \{\{\},\{a\},\{b\},\{c\},\{a,b\},\{a,c\},\{b,c\},\{a,b,c\}\}\)

\(P(A -\{a\}) = P(\{b,c\}) = \{\{\},\{b\},\{c\},\{b,c\}\}\)

\subsection{}
\(A:= \{z \cup \{x\} : z \in PT\}\)

\(P(S)= PT \cup A\)

\section{Problem 4}

\subsection{DLN 2.124}

\(A,B = 2\)

\(A,C = 1\)

\(B,C = 1\)

\subsection{DLN 2.125}

\(A,B = \frac{2}{5}\)

\(A,C = \frac{1}{3}\)

\(B,C = \frac{1}{4}\)

\subsection{DLN 2.126}

This statement is not true for the cardinality measure. A simple example to disprove it would be the sets \(A = \{1,2,3\}\), \(B = \{1,2,3,4,5,6,7\}\) and \(C = \{4,5,6,7\}\). The cardinality measure for $A$ and $B$ would be $3$, and $0$ for $A$ and $C$. However, the highest cardinality measure for $B$ would be between $B$ and $C$, equal to $4$. Thus, $A$ is most similar to $B$, but $B$ is most similar to $C$.

\subsection{DLN 2.127}

This statement is also not true for the Jaccard coefficient. The same set of sets described above disprove it. For $A$ and $B$, the Jaccard coefficient would be \(\frac{3}{7}\), and \(0\) for $A$ and $C$. The Jaccard coefficient for $B$ and $C$ would be \(\frac{4}{7}\). Thus, $A$ is most similar to $B$, but $B$ is most similar to $C$.


\end{document}