\documentclass[titlepage]{article}
\usepackage{hw}  % leave this line

\psetAuthor{Caio Brighenti}  % replace with your name
\psetNumber{8}  % replace with pset number

\begin{document} \maketitle

\section{Problem 1: Counting alignments}

\subsection{DLN 9.153}

For this problem, the order is irrelevant, and repetition is allowed, thus the general formula is ${n+k-1}\choose{k}$. In the case, $n=2n$, and $k=n$. Therefore, we have that ${3n-1}\choose{n}$.

\subsection{DLN 9.154}

For this problem, the order is irrelevant, and repetition is not allowed, thus the general formula is ${n}\choose{k}$. In the case, $n=2n$, and $k=n$. Therefore, we have that ${2n}\choose{n}$.

\subsection{DLN 9.155}

For this problem, the order matters, and repetition is allowed, thus the general formula is $n^k$. In the case, $n=2n$, and $k=n$. Therefore, we have that ${2n}^n$.

\subsection{DLN 9.156}

For this problem, the order matters, and repetition is not allowed, thus the general formula is $\frac{n!}{(n-k)!}$. In the case, $n=2n$, and $k=n$. Therefore, we have that $\frac{2n!}{n!}$.

\section{Problem 2: Counting bit strings, DLN 9.166}

\textbf{Claim}: Let $S_{n,k}$ be the set of all bitstrings with $n$ zeroes and $k$ ones without any adjacent ones. Let $P(n)$ if $|S_{n,k}| = {{n+1}\choose{k}}$.  The claim is that $\forall n \in \mathrm{Z}^{\geq 0}: P(n)$.

\begin{proof}
We will prove by weak induction on $n$.

\textbf{Base cases}: 
\begin{itemize}
\item For any value of $n$ (including the cases below), there is the case that $k > n+1$. By the definition of the binomial coeficient, in this case ${n+1}\choose{k}$ will always equal $0$. This fits the claim, as if $k > n+1$ we have $S_{n,k} = \emptyset$.
\item \emph{$n = 0$}: If we have no zeroes, we have the cases $k \in \{0,1\}$. For $k=1$, the only bitstring will be $\langle 1 \rangle$. Since ${{0+1}\choose{1}}=1$, this holds. For $k=0$, we have ${{0}\choose{0}}=1$. This is true, as the only valid bitstring with no zeroes or ones will be $\emptyset$. Thus, $S_{0,0} = \{\emptyset\}$ and therefore $|S_{0,0}|=1$, giving us $P(0)$.
\item \emph{$n = 1$}: For a single zero, we have the cases $k \in \{0,1,2\}$. If $k=0$, the only bit string will be $\langle 0 \rangle$, and since ${{1+1}\choose{0}}=1$ the claim holds for $k=0$. For $k=1$, the valid bit strings will be $\{\langle 0,1 \rangle , \langle 1,0 \rangle\}$, and since ${{1+1}\choose{1}}=2$, the claim holds for $k=1$. Finally, when $k=2$, we have the possible bit string $\langle 1,0,1 \rangle$, and since ${{1+1}\choose{2}}=1$, the claim holds for $k=2$. Thus, we have that $P(1)$. 
\end{itemize}


\textbf{Inductive case}: Let $n \geq 2$.  We will show $ P(n-1) \implies P(n)$.
\begin{itemize}
\item \emph{Given}: Assume $P(n-1)$ is true.
\item \emph{Want to show}: $P(n)$ is true.
\end{itemize}
If the number of zeroes is $n$, and the number of ones is $k$, the length of the resulting bitstring must be $n+k$. This bitstring can exist in two ways: either beginning with a $0$ or a $1$. In the case that the bitstring begins with a $0$, the length of the remaining bitstring will be equal to $n-1+k$. We will call the set of all possible bitstrings in this case $A$. This case is illustrated below.
\begin{center}
  \begin{tabular}{ | c | c |}
    \hline
    $0$ & $n-1+k$  \\ \hline
  \end{tabular}
\end{center}
Alternatively, if the bitstring begins with $1$, the following digit must be $0$ according to the rule that no two $1$s may be adjacent. From there, the length of the remaining bitstring would be $n-1+k-1$. We will call the set of all possible bitstrings in this case $B$. This case is illustrated below:
\begin{center}
  \begin{tabular}{ | c | c | c |}
    \hline
    $1$ & $0$ & $n-1+k-1$  \\ \hline
  \end{tabular}
\end{center}
As $A$ represents all valid bitstrings of length $n+k$ starting with a 0, and $B$ represents those starting with $1$, the two sets are disjoint. Aditionally, it holds that $S_{n,k} = A+B$. By the inductive hypothesis, we have that $P(n-1)$, thus $|A|={{n}\choose{k}}$ and that $|B|={{n}\choose{k-1}}$. We will use these facts to prove that $P(n)$.
\begin{align*}
|S_{n,k}| &= |A|+|B| & \text{sum rule for disjoint sets} \\
&= {{n}\choose{k}}+{{n}\choose{k-1}} & \text{inductive hypothesis} \\
&= {{n+1}\choose{k}} & \text{Pascal's rule}\\
\end{align*}
Thus, the claim is true.
\end{proof}

\section{Problem 3: Combinatorial proof DLN 9.168}

\textbf{Claim}: The claim is that $k \cdot {{n}\choose{k}} = n \cdot {{n-1}\choose{k-1}}$. 
\begin{proof} We will first prove the claim algebraically, then provide a combinatorial proof to support the algebra.
\begin{align*}
n \cdot {{n-1}\choose{k-1}} &= n \cdot \frac{(n-1)!}{(k-1)!(n-k)!} & \text{definition of binomial coeficient} \\
&= \frac{n(n-1)!}{(k-1)!(n-k)!} & \text{rearranging} \\
&= \frac{n!}{(k-1)!(n-k)!} & n(n-1)! = n \\
&= \frac{k \cdot n!}{k(k-1)!(n-k)!} & \text{multiply by $\frac{k}{k}$} \\
&= k \cdot \frac{n!}{k!(n-k)!} & k(k-1)! = k \\
&= k \cdot {{n}\choose{k}} & \text{definition of binomial coeficient} \\
\end{align*}
Thus, the claim is true algebraically. We will now show the same combinatorially. \\
A person is packing for a business trip with a flight leaving later that day. They can only take $k$ number of shirts with them, which must be selected from their $n$ number of shirts. After selecting all their shirts, they must select one of these to wear during the flight. The number of choices of which shirts to take can be represented as ${n}\choose{k}$. After selecting $k$ number of shirts, the number of options for the shirt to wear is $k$. Thus, the total amount of choices is $k \cdot {{n}\choose{k}}$.\\
Alternatively, the person could have first chosen what they were going to wear, before packing the rest of their shirts. Thus, the choice for what shirt to wear would be $n$, as they have $n$ shirts. From there, they have $n-1$ shirts left to choose from, and can take $k-1$ more shirts. Their choices would be ${n-1}\choose{k-1}$, making their total choices equal to $n \cdot {{n-1}\choose{k-1}}$. \\
Since the two scenarios are equal, the number of choices in each is equal, and thus the claim is true.
\end{proof}
\end{document}