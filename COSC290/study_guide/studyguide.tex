\documentclass[titlepage]{article}
\usepackage{hw}  % leave this line

\psetAuthor{Caio Brighenti}  % replace with your name
\psetNumber{Final Study Guide}  % replace with pset number

\begin{document} \maketitle

\section{Data Structures}
\label{sec:introduction}

\subsection{Sets}
\emph{Definition: }A set is an unordered collection of objects.
\begin{itemize}
  \item  $Fruits := \{banana, apple, pear\}$
  \item Membership: $apple \in Fruits$ is True
  \item Subset: $\{ banana, pear \} \subset Fruits$
  \item Cardinality: $|Fruits| = 3$
  \item Enumeration: $SingleDigitOdds := \{1, 3, 5, 7, 9\}$
  \item Abstraction: $SingleDigitOdds := \{ 2x + 1 : x \in Z$ and $0 \leq x \leq 4 \}$
\end{itemize}
\emph{Powerset: }The powerset of a set $S$ is the set of all subsets of $S$

\subsection{Sequences}
\emph{Definition: }A sequence is an ordered collection of objects. (Also known as lists
or tuples.)
\begin{itemize}
  \item Example: MWF course schedule $\langle$COSC 290, Math 260, CORE 151 D$\rangle$.
  \item Order matters, can have duplicates, represented using angle brackets.
\end{itemize}
\textbf{Cartesian Product: } Takes two sets and generates a set of ordered pairs (sequences of length two).
\begin{itemize}
  \item $A \bigtimes B := \{ \langle a, b \rangle : a \in A$ and $b \in B \}$
  \item $A \bigtimes B \bigtimes C := \{ \langle a, b, c\rangle : a \in A$ and $b \in B$ and $c \in C \}$
\end{itemize}
\subsection{Functions}
\emph{Definition: }A function $f$ from $X$ to $Y$, written $f : X \rightarrow Y$, assigns each input value
$x \in X$ to a unique output value $y \in Y$.
\begin{itemize}
  \item All $x \in X$ must be mapped to a $y$, and no $x$ can be mapped to more than one $y$.
  \item One-to-one: for every $y \in Y$, there is at most one $x \in X$ such that $f(x) = y$.
  \item Onto: for every $y \in Y$, there is at least one $x \in X$ such that $f(x) = y$.
  \item Bijective: both one-to-one and onto.
\end{itemize}

\section{Logic}
\subsection{Propositions}
\emph{Definition: }A proposition is a sentence that is either true or false.
\begin{itemize}
  \item 3 + 4 = 6
  \item My middle name is Gerald or my dog's name is Rufus.
  \item A single proposition is called an \textbf{atomic proposition}, while a \textbf{compound proposition} is made up of several atomic propositions.\\
\end{itemize}
\begin{tabular}{c|c|c}
\textbf{$Name$} & \textbf{$Symbol$} & \textbf{$Meaning$} \\ \hline
Negation & $\lnot p$ & "not p" \\
Conjunction & $p \land q$ & "p and q" \\
Disjunction & $p \lor q$ & "p or q" \\
Exclusive or & $p \oplus q$ & "p or q, but not both" \\
Implies & $p \implies q$ & "if p, then q" \\
Mutual implication & $p \iff q$ & "p if and only q" \\
\end{tabular}
\\
\\Two sentences are \textbf{logically equivalent} if they have identical truth tables.  
\\A proposition is a \textbf{tautology} if it is true under every assignment of its variables. 
\\A \textbf{literal} is an atomic proposition or the negation of an atomic proposition (either $p$ or $\lnot p$ for some variable $p$).
\section{Counting and Combinations}

\section{Probability}

\end{document}