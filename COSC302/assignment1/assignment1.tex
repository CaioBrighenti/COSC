\documentclass{article}
\usepackage{amsmath} %This allows me to use the align functionality.
                     %If you find yourself trying to replicate
                     %something you found online, ensure you're
                     %loading the necessary packages!
\usepackage{amsfonts}%Math font
\usepackage{graphicx}%For including graphics
\usepackage{hyperref}%For Hyperlinks
\usepackage{listings}
\usepackage{graphicx}
\usepackage{natbib}        %For the bibliography
\bibliographystyle{apalike}%For the bibliography
\usepackage[margin=1.0in]{geometry}
\usepackage{float}
\begin{document}
%set the size of the graphs to fit nicely on a 8.5x11 sheet
\noindent \textbf{Caio Brighenti }\\
\noindent \textbf{COSC 302 - Analysis of Algorithms -- Spring 2019}\\%\\ gives you a new line
\noindent \textbf{Assignment 1}\vspace{1em}\\
\begin{enumerate}
\end{enumerate}
	\item \\
	Simply put, an algorithm is a way to solve a problem. In other words, it is a specific, well-defined step-by-strep process to solve not just a single problem, but a category of similar problems. Algorithms must take some form of input, and must correctly use this input to solve its defined category of problems.
	\item \\
	

\end{document}
