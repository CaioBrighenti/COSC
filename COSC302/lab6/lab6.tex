\documentclass{article}
\usepackage{amsmath} %This allows me to use the align functionality.
                     %If you find yourself trying to replicate
                     %something you found online, ensure you're
                     %loading the necessary packages!
\usepackage{amsfonts}%Math font
\usepackage{graphicx}%For including graphics
\usepackage{hyperref}%For Hyperlinks
\usepackage{listings}
\usepackage{graphicx}
\usepackage{natbib}        %For the bibliography
\bibliographystyle{apalike}%For the bibliography
\usepackage[margin=1.0in]{geometry}
\usepackage{float}
\usepackage{tikz}
\usetikzlibrary{trees}
\begin{document}
%set the size of the graphs to fit nicely on a 8.5x11 sheet
\noindent \textbf{Caio Brighenti }\\
\noindent \textbf{COSC 302 - Analysis of Algorithms -- Spring 2019}\\%\\ gives you a new line
\noindent \textbf{Lab 5}\vspace{1em}\\
\begin{enumerate}
	\setcounter{enumi}{8}
	\item Recursion tree method.
	\\ The recurrence is $T(n)=T(n-a)+T(a)+cn$ for all $a\geq1$ and $c>0$. Thus, the recursion tree is: \\
\begin{center}
	\begin{tikzpicture}[level distance=1.5cm,
	  level 1/.style={sibling distance=3cm},
	  level 2/.style={sibling distance=1.5cm}]
	  \node {T(n)}
	    child {node {T(n-a)}
	      	child {node {T(n-2a)}
	      		child {node {T(n-3a)}
	      		}
	      		child {node {T(a)}
	      		}
	      	}
	      	child {node {T(a)}
	      	}
	    }
	    child {node {T(a)}
		};
	\end{tikzpicture}
\end{center}
Let the first level of the tree ($T(n)$) be $i=0$, and each subsequent level sum one to $i$. Then, we see that the leftmost leaf in each level is equal to $T(n-ia)$. We assume the base case of the algorithm happens at 0, thus lowest level of the tree will be $T(1)$. Using this, we can solve for the total number of levels, $i$, as follows.
\begin{align}
	n-ia&=1 && \text{} \\
	ia&=n-1 && \text{} \\
	i&=\frac{n-1}{a} && \text{}
\end{align} 
Thus, the total number of levels $i$ is $\frac{n-1}{a}$. We must now find the running time per call. We have this information from the recurrence. Each call takes $T(a)+cn$ time, but as the linear term dominates, then the running time is $\Theta (n)$. Thus, the solution to the recurrence is as follows:
$$(\frac{n-1}{a})(cn)=\Theta(n)\Theta(n)\in \Theta(n^2)$$
As we have a factor $n$ multiplied by another factor $n$. Each other element is a constant.
\end{enumerate}


\end{document}
