\documentclass{article}
\usepackage{amsmath} %This allows me to use the align functionality.
                     %If you find yourself trying to replicate
                     %something you found online, ensure you're
                     %loading the necessary packages!
\usepackage{amsfonts}%Math font
\usepackage{graphicx}%For including graphics
\usepackage{hyperref}%For Hyperlinks
\usepackage{listings}
\usepackage{graphicx}
\usepackage{natbib}        %For the bibliography
\bibliographystyle{apalike}%For the bibliography
\usepackage[margin=1.0in]{geometry}
\usepackage{float}
\begin{document}
%set the size of the graphs to fit nicely on a 8.5x11 sheet
\noindent \textbf{Caio Brighenti }\\
\noindent \textbf{COSC 302 - Analysis of Algorithms -- Spring 2019}\\%\\ gives you a new line
\noindent \textbf{Lab 2}\vspace{1em}\\
\begin{enumerate}
\setcounter{enumi}{4}
\item Find a constant $c<1$ such that $F_n \leq 2^{cn}$ for all $n\geq 0$.
\\ We proceed by induction. We select the constant $c=0.9$. First, we show that the base cases are correct. As this problem pertains to the Fibonacci sequence, our base cases are $n=0$ and $n=1$. 
\\ \textbf{Base case 1 - $n=0$}\\
$F_0=0\leq 2^{0.9\cdot0}=1$ \\
As the expression with the constant evaluates to 1, and we know $F_0$ to be 0, the first base case holds.
\\ \textbf{Base case 2 - $n-1$}\\
$F_1=1\leq 2^{0.9\cdot1}\approx1.86$ \\
As the expression with the constant evaluates to approximately 1.86, and we know $F_1$ to be 1, the second base case holds.
\\
\\ \textbf{Inductive hypothesis}
\\ Through the inductive hypothesis, we assume that the claim is true for all values from $0$ to $n+2$. In other words, we assume the claim is true for $n$ and $n+1$, and we proceed by showing that it is thus true for $n+2$.
\begin{align}
	F_{n+2}&=F_{n}+F_{n+1} && \text{definition of Fibonacci} \\
	&\leq 2^{0.9n}+2^{0.9n+0.9} && \text{inductive hypothesis} \\
	&\leq 2^{0.9n}(1+2^{0.9}) && \text{common factor} \\
	&\leq \frac{2^{1.8}(2^{0.9n}(1+2^{0.9}))}{2^{1.8}} && \text{multiply by 1, note 1.8 = 2c} \\
	&\leq 2^{0.9(n+2)}\frac{(1+2^{0.9})}{2^1.8}\approx 0.82(2^{0.9(n+2)}) && \text{exponent rules}	
\end{align}
Thus, we have shown that: \\
$F_{n+2}\approx 0.82(2^{0.9(n+2)})$ \\
and therefore that: \\
$F_{n+2}\leq2^{c(n+2)}$ \\
where $c=0.9$. Thus, the claim is true for a constant $c$ of $0.9$.
\end{enumerate}


\end{document}
