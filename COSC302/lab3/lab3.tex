\documentclass{article}
\usepackage{amsmath} %This allows me to use the align functionality.
                     %If you find yourself trying to replicate
                     %something you found online, ensure you're
                     %loading the necessary packages!
\usepackage{amsfonts}%Math font
\usepackage{graphicx}%For including graphics
\usepackage{hyperref}%For Hyperlinks
\usepackage{listings}
\usepackage{graphicx}
\usepackage{natbib}        %For the bibliography
\bibliographystyle{apalike}%For the bibliography
\usepackage[margin=1.0in]{geometry}
\usepackage{float}
\begin{document}
%set the size of the graphs to fit nicely on a 8.5x11 sheet
\noindent \textbf{Caio Brighenti }\\
\noindent \textbf{COSC 302 - Analysis of Algorithms -- Spring 2019}\\%\\ gives you a new line
\noindent \textbf{Lab 2}\vspace{1em}\\
My assigned problem was \#1. Problems 6,7, and 9 are my other three problems completed.
\begin{enumerate}
	\item \textbf{Prove or disprove each of the following:}
	\begin{enumerate}
		\item $f(n)=O(g(n)) \implies g(n)=O(f(n))$ \\
			\\ We disprove by counterexample. Let $f(n)=n$ and $g(n)=n^2$.  $n=O(n^2)$, but $n^2\neq O(n)$. Thus, the claim is false. \\
		\item $f(n)+g(n)=\Theta (min(f(n),g(n)))$ \\ 
			\\ We disprove by counterexample. Let $f(n)=n$ and $g(n)=n^2$. Then $n^2+n\neq \Theta (n)$, therefore the claim is false. \\
		\item $f(n)=O(g(n))\implies log(f(n))=O(log(g(n)))$ \\
			\\ We proceed by direct proof. 
			\begin{align}
				f(n)&\leq c_2 g(n) && \text{definition of O(g(n)} \\
				log(f(n))&\leq log(c_2 g(n)) && \text{log both sides} \\
				&\leq log(c_2)+log(g(n))&& \text{log properties} 
			\end{align}
			Let $c_3=log(c_2)$. Thus, \\
			$log(f(n))\leq c_3 log(g(n))$ \\
			As this is precisely the definition of $log(f(n))=O(log(g(n)))$, the claim must be true.
	\end{enumerate}
	\setcounter{enumi}{5}
	\item \textbf{Give definitions for $\Omega(g(n,m))$ and $\Theta(g(n,m))$} \\
		\\ $\Omega(g(n,m))=\{ f(n,m): $ there exists positive constants $c_2$, $n_0$, and $m_0$
		\\ such that $f(n,m)\leq c_2(g(n,m))$
		\\ for all $n\geq n_0$ and $m\geq m_0\}$. 
		\\ \\ $\Theta(g(n,m))=\{ f(n,m): $ there exists positive constants $c_1$, $c_2$, $n_0$, and $m_0$
		\\ such that $c_1(g(n,m))\leq f(n,m)\leq c_2(g(n,m))$
		\\ for all $n\geq n_0$ and $m\geq m_0\}$. 
	\item \textbf{Is $2^{n+1}=O(2^n)$? Is $2^{2n}=O(2^n)$?}
		\\ \\ We have $2^{n+1}=2\cdot2^n$. As $2\cdot 2^n = \Theta(2^n)$, it follows that $2^{n+1} = \Theta(2^n)$.
		\\ The second case is not true. We proceed by contradiction. Assume that $2^{2n}=O(2^n)$. We must thus have that:
		\\ $2^{2n}\leq c_2 2^n$ by the definition of $\Theta(g(n))$. By rearranging, we have that:
		\\ $\frac{2^{2n}}{2^n}\leq c_2 $ and finally that:
		\\ $2^n \leq c_2 $.
		\\ This statement is impossible, as a constant cannot be greater than or equal to a variable that grows to infinity. Thus, we have a contradiction, and the claim cannot be true.
	\setcounter{enumi}{5}
	\item \textbf{Let $f(n)$ and $g(n)$ be asymptotically non-negative functions. Using the basic definition of $\Theta$-notation, prove  that $max(f(n),g(n))= \Theta(f(n)+g(n))$.} \\
	\\ As $f(n)$ and $g(n)$ are both asymptotically nonnegative, then for all $n>n_0$ it must be that $f(n)>0$ and $g(n)>0$. Thus, we must have that:
	\\ $f(n)+g(n)\geq f(n) \geq 0$ and that $f(n)+g(n)\geq g(n) \geq 0$.
	\\ Let $h(n)=max(f(n),g(n))$.
	\\ As $h(n)$ will always equal the biggest of the two functions, it is true that $h(n)\geq f(n)$ and $h(n)\geq g(n)$. Thus, we can substitute $h(n)$ and obtain the following:
	\\ $$f(n)+g(n)\geq h(n) \geq 0$$
	\\ If we add a constant $c_2$, we can have that:
	\\ $$h(n)=max(f(n),g(n))\leq c_2(f(n)+g(n))$$ 
	\\ Thus, we have $max(f(n),g(n))= O(f(n)+g(n))$.
	\\ We also can write that $0 \leq f(n) \leq h(n)$ and $0 \leq g(n) \leq h(n)$. Adding these two inequalities results in:
	\\ $$0\leq f(n)+g(n)\leq 2h(n)$$
	\\ Thus, by rearranging this we have:
	\\ $$h(n)=max(f(n),g(n))\geq c_1(f(n)+g(n))$$
	\\ Thus, $max(f(n),g(n))= \Omega(f(n)+g(n))$.
	\\ Since we have that $max(f(n),g(n))= O(f(n)+g(n))$ and that $max(f(n),g(n))= \Omega(f(n)+g(n))$, it must be that $max(f(n),g(n))= \Theta(n)+g(n))$.
\end{enumerate}


\end{document}
