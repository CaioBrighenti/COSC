\documentclass{article}
\usepackage{amsmath} %This allows me to use the align functionality.
                     %If you find yourself trying to replicate
                     %something you found online, ensure you're
                     %loading the necessary packages!
\usepackage{amsfonts}%Math font
\usepackage{graphicx}%For including graphics
\usepackage{hyperref}%For Hyperlinks
\usepackage{listings}
\usepackage{graphicx}
\usepackage{natbib}        %For the bibliography
\bibliographystyle{apalike}%For the bibliography
\usepackage[margin=1.0in]{geometry}
\usepackage{float}
\usepackage{tikz}
\usetikzlibrary{trees}
\begin{document}
%set the size of the graphs to fit nicely on a 8.5x11 sheet
\noindent \textbf{Caio Brighenti }\\
\noindent \textbf{COSC 302 - Analysis of Algorithms -- Spring 2019}\\%\\ gives you a new line
\noindent \textbf{Lab 5}\vspace{1em}\\
\begin{enumerate}
	\setcounter{enumi}{2}
	\item Recursion tree method.
	\\ The recurrence is $T(n)=T(n-1)+T(n-2)$. Thus, the recursion tree is: \\
\begin{center}
	\begin{tikzpicture}[level distance=1.5cm,
	  level 1/.style={sibling distance=3cm},
	  level 2/.style={sibling distance=1.5cm}]
	  \node {n}
	    child {node {n-1}
	      child {node {n-2}}
	      child {node {n-3}}
	    }
	    child {node {n-2}
	    child {node {n-3}}
	      child {node {n-4}}
	    };
	\end{tikzpicture}
\end{center}
Note that this tree is not balanced, but we will assume it is since we are showing an upper bound with $O(f(n))$ notation. We can see that the slowest decreasing element of the tree, the leftmost node, decreases by 1 on each level. We will call the total number of levels $i$. We assume the function recurses until 0 -- therefore since $n$ decreases by 1 each level, we will have $n$ levels, thus $i=n$.
\\ Next, we find the number of nodes per level. We can see that the first level has $2^0$ nodes, the second $2^1$ nodes, and the third $2^2$. Thus, if we let $j$ represent the current level, then the number of nodes for each level is equal to $2^{(j-1)}$. Thus, we can represent the total amount of nodes (or calls) as, since we know we have $i$ levels:
$$2^0+2^1+2^2 + \cdots + 2^{i-1}$$
This can be simplified to $2^i-1=2^n-1$. Thus, we solution to the recurrence is:
$$T(n)=2^n-1=O(2^n)$$
\end{enumerate}


\end{document}
