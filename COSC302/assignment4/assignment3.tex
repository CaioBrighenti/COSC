\documentclass{article}
\usepackage{amsmath} %This allows me to use the align functionality.
                     %If you find yourself trying to replicate
                     %something you found online, ensure you're
                     %loading the necessary packages!
\usepackage{amsfonts}%Math font
\usepackage{graphicx}%For including graphics
\usepackage{hyperref}%For Hyperlinks
\usepackage{listings}
\lstset{
    numbers=left,
    backgroundcolor = \color{lightgray},
    breaklines=true,
    tabsize=2,
    basicstyle=\ttfamily,
    literate={\ \ }{{\ }}1
}
\usepackage{graphicx}
\usepackage{fancyvrb}
\usepackage{natbib}        %For the bibliography
\bibliographystyle{apalike}%For the bibliography
\usepackage[margin=1.0in]{geometry}
\usepackage{float}
\usepackage{tikz}
\usetikzlibrary{trees}
\begin{document}
%set the size of the graphs to fit nicely on a 8.5x11 sheet
\noindent \textbf{Caio Brighenti }\\
\noindent \textbf{COSC 302 - Analysis of Algorithms -- Spring 2019}\\%\\ gives you a new line
\noindent \textbf{Assignment 4}\vspace{1em}\\
\begin{enumerate}
	\item \textbf{Articulation points}
	\begin{enumerate}
		\item Give an algorithm to find if graph $G$ is connected
		\\ A connected graph $G$ is a graph for which there exists a between every pair of vertices in the graph. That is to say, every vertex is reachable from a traversal starting at any arbitrary vertex. We leverage this fact to devise a DFS-based algorithm to determine whether a graph $G$ is connected.
		\\ \textbf{Algorithm:}
		\begin{lstlisting}
			isConnected(G){
				v = any vertex in G
				explore(G,v)			
			}
			
			explore(G,v){
				visited(v) = true
				add v to seen[]
				for each edge(v,u)	
			}
		\end{lstlisting}
	\end{enumerate}
\end{enumerate}
	

\end{document}
