\documentclass{article}
\usepackage{amsmath} %This allows me to use the align functionality.
                     %If you find yourself trying to replicate
                     %something you found online, ensure you're
                     %loading the necessary packages!
\usepackage{amsfonts}%Math font
\usepackage{graphicx}%For including graphics
\usepackage{hyperref}%For Hyperlinks
\usepackage{listings}
\usepackage{graphicx}
\usepackage{natbib}        %For the bibliography
\bibliographystyle{apalike}%For the bibliography
\usepackage[margin=1.0in]{geometry}
\usepackage{float}
\begin{document}
%set the size of the graphs to fit nicely on a 8.5x11 sheet
\noindent \textbf{Caio Brighenti }\\
\noindent \textbf{COSC 480 - Learning From Data}\\%\\ gives you a new line
\noindent \textbf{Fall 2019}\\%\\ gives you a new line
\noindent \textbf{Problem Set 6}\vspace{1em}\\
\begin{enumerate}
	\item Problem 2.3
	\begin{enumerate}
		\item Positive or negative ray
		\\\\ Given a hypothesis set consisting of either a positive or negative ray, the total number of dichotomies will be equal to $|D_1 \cup D_2|$, where $D_1$ is the dichotomy set for positive rays, and $D_2$ is the dichotomy set for negative rays. As defined earlier in the chapter, the number of dichotomies for a ray is $N+1$. Thus, we have that $|D_1 \cup D_2| = (N+1)+(N+1)-|D_1 \cap D_2|$. The set $D_1 \cap D_2$ clearly will only have two elements: the dichotomy when $h$ for all values in $X$ is $+1$, and when $h$ is $-1$ for all $x \in X$.
		\\ Therefore, the total number of dichotomies is $m_h(N) = |D_1 \cup D_2| = (N+1)+(N+1)-2=2N$.
		\item Positive or negative interval
		\\\\ We approach this problem in the same manner as the one before. The total number of dichotomies is equal to $|D_1 \cup D_2| = |D_1| + |D_2| - |D_1 \cap D_2|$, where $D_1$ is the set dichotomies for positive intervals and $D_2$ is the set of dichotomies for negative intervals. In both the negative and positive case, the maximum number of dichotomies is $m_h(N)=\frac{1}{2}N^2+\frac{1}{2}N+1$. It remains to be shown what the set $D_1 \cap D_2$ is composed of.
		\\ The only cases where a positive and negative interval overlap is when at least one endpoint of the interval is at $x_1$ or $x_n$. In any of these cases, the intervals that produce the same dichotomy set across positive and negative intervals can be expressed by $[x_1,a]$ and $[a,x_n]$, where one of these is the endpoints of a positive interval and the other a negative interval. Thus we must find the number of possible intervals with at least one endpoint equal to $x_1$ or $x_n$.
		\\ This is clearly the same as the case above, where we select an endpoint $a$ and the second endpoint from either $x_1$ or $x_n$. Thus, $|D_1 \cap D_2| = 2N$, and therefore $m_h(N)=|D_1 \cup D_2|=\frac{1}{2}N^2+\frac{1}{2}N+1+\frac{1}{2}N^2+\frac{1}{2}N+1-2N=N^2 + N + 2 - 2N = N^2 - N + 2$.
	\item Two cocentric spheres
	\\\\ While this problem appears highly complicated at first, it is simply a specialized version of the interval problem. Each dichotomy is expressed as the radius of a sphere, cocentric with the spheres of radius $a$ and $b$. A radius is $+1$ if it falls between the range $a,b$, and $-1$ otherwise. This is clearly just an interval on the radii produced by $X$, and thus the maximum number of dichotomies is simply $\frac{1}{2}N^2 + \frac{1}{2}N +1$.
	\end{enumerate}
\end{enumerate}
	

\end{document}
