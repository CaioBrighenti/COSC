\documentclass{article}
\usepackage{amsmath} %This allows me to use the align functionality.
                     %If you find yourself trying to replicate
                     %something you found online, ensure you're
                     %loading the necessary packages!
\usepackage{amsfonts}%Math font
\usepackage{graphicx}%For including graphics
\usepackage{hyperref}%For Hyperlinks
\usepackage{listings}
\usepackage{graphicx}
\usepackage{natbib}        %For the bibliography
\bibliographystyle{apalike}%For the bibliography
\usepackage[margin=1.0in]{geometry}
\usepackage{float}
\begin{document}
%set the size of the graphs to fit nicely on a 8.5x11 sheet
\noindent \textbf{Caio Brighenti }\\
\noindent \textbf{COSC 480 - Learning From Data -- Fall 2019}\\%\\ gives you a new line
\noindent \textbf{Problem Set 1}\vspace{1em}\\
\begin{enumerate}
	\item What makes a problem well-suited for machine learning?
	\\\\ A problem is suitable for machine learning when a design or analytical solution is not possible, and there is readily available data from which a machine learning solution can be derived. 
	\item What are the components of learning?
	\\\\ We learn from data using the dataset $D$ contained $N$ observations in the input space $X$ and $Y$ such that each observation has $N$ $(x,y)$ pairs. We use this data to pick a candidate function $g: X \rightarrow Y$ from the hypothesis set $H$ that best approximates $f: X \rightarrow Y$. 
	\item What must be true about your training data for the PLA algorithm to terminate?
	\\\\ The PLA algorithm learns iteratively until the perceptron is able to classify all observations correctly. However, in the case that the training data is not linearly seperable (i.e. there is no linear function that can perfectly separate the classes), the algorithm will run infinitely and never converge.
\end{enumerate}
	

\end{document}
